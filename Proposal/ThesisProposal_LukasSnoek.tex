%%%%%%%%%%%%%%%%%%%%%%%%%%%%%%%%%%%%%%%%%%%%%%%%%%%%%%%%%%%%%%%%
% Latex template for thesis proposal form from 
% the University of Amsterdam. 
% 
% The title-page template was downloaded from 
% latextemplates.com and adapted to fit the
% RMP thesis proposal form. 
%
% Original author title-page template:
% WikiBooks (http://en.wikibooks.org/wiki/LaTeX/Title_Creation)
% CC BY-NC-SA 3.0 (http://creativecommons.org/licenses/by-nc-sa/3.0/)
%
%%%%%%%%%%%%%%%%%%%%%%%%%%%%%%%%%%%%%%%%%%%%%%%%%%%%%%%%%%%%%%%%

%%%%%%%%%%%%%%%%%%%%%%%%%%%%%%%%%%%%%%%%%%%%%%%%%%%%%%%%%%%%%%%%
% START PRE-AMBLE
\documentclass[12pt,a4paper]{article}
\usepackage[utf8]{inputenc}
\usepackage{amssymb} 
\usepackage{natbib}
\bibliographystyle{apalike}

\usepackage{setspace} 
\setlength{\parindent}{5mm}

\usepackage{graphicx}
\graphicspath{ {images/} }

% Header style
\usepackage{fancyhdr}
\pagestyle{fancy}
\fancyhf{}
\renewcommand{\headrulewidth}{0pt}
\lfoot{Graduate School of Psychology}
\rfoot{Page \thepage}

\begin{document}

%%%%%%%%%%%%%%%%%%%%%%%%%%%%%%%%%%%%%%%%%%%%%%%%%%%%%%%%%%%%%%%%
% Start titlepage environment
\begin{titlepage}

\newcommand{\HRule}{\rule{\linewidth}{0.5mm}} 
\center 

% Headings
\textsc{\LARGE Graduate School of Psychology}\\[1cm] 
\textsc{\Large University of Amsterdam}\\[1cm] % Major heading such as course 

% Title section
\HRule \\[0.4cm]
{ \huge \bfseries Research Master's Psychology Thesis Proposal Form}\\[0.4cm] 
\HRule \\[1.5cm]
 
% Author section & version/data info
\begin{minipage}{0.4\textwidth}
\begin{flushleft} \large
\emph{Author:}\\
Lukas \textsc{Snoek} 
\end{flushleft}
\end{minipage}
~
\begin{minipage}{0.4\textwidth}
\begin{flushright} \large
\emph{Version:} \\
\textsc{1st draft} 
\end{flushright}
\end{minipage}\\[2cm]

% Logo
\includegraphics[scale=0.15]{uva_logo3}\\[2cm] 

% Data
{\large \today}\\[3cm]

\vfill 

\end{titlepage}

%%%%%%%%%%%%%%%%%%%%%%%%%%%%%%%%%%%%%%%%
% Begin actual proposal form
\onehalfspacing

\section{Who and Where?}
\vspace{\baselineskip}

\begin{tabular}{l l l}
\textbf{Student} \\
Name & : &                              Lukas Snoek\\
Student ID number & : &                 10126228 \\
Address & : &                           Commelinstraat 332 \\
Postal code and residence & : &         1093VD, Amsterdam \\
Telephone number & : &                  +31647769183 \\
Email address & : &                     lukassnoek@gmail.com \\[0.5cm]

\textbf{Supervisor(s)} \\
Within ResMas (obligatory) & : &        Dr. H.S. Scholte \\
Specialisation & : &                    Brain \& Cognition \\
External supervisor(s), if any & : &    n/a \\
Second assessor & : &                   Prof. dr. R. Ridderinkhof \\
Research center / location & : &        Spinoza centre for neuroimaging \\[0.5cm]

\textbf{Number of credits} & : &        40 \\
\textit{At least 25 ec} \\[0.5cm]

\textbf{Ethics committee reference code} & : &   tbd \\
\textit{https://www.lab.uva.nl/ce/} \\[0.5cm]

\end{tabular}

%%%%%%%%%%%%%%%%%%%%%%%%%%%%%%%%%%%%%%%%
\section{Title and summary research project}

\subsection{Title}
Investigating the dimensionality and spatial distribution of functional brain networks 

\subsection{Summary of proposal \textmd{- max 500 words}}
In the past decade, cognitive neuroscience witnessed a paradigm shift from functional localization to distributed representations of psychological processes in the brain. New analytical tools were developed to fill this methodological niche, including functional connectivity and multivariate modeling. Especially the latter technique, often known as Multivariate Pattern Analysis (MVPA), has shown to be a promising method to characterize and investigate distributed functional networks in the brain . Studies employing these multivariate analyses often claim to reveal a \emph{distributed} neural network representing a certain psychological process. In the psychological literature, however, the term "distributeness" has been used ambiguously and inconsistently, as multivariate representations may be present within a cluster of voxels, within a cortical lobe, or even across the entire brain \citep{coutanche2013}. For example, while object categories may be represented throughout the anterior temporal lobe \citep{haxby2001}, emotion networks seem to be represented across the entire brain \citep{lindquist2012}. Furthermore, given the spatial distribution of a psychological process, the question remains whether the representation is encoded within a distributed set of \emph{voxels} or within a distributed set of \emph{brain areas}. This study's proposal aims to investigate the spatial distribution and multivariate structure of two types of processes: perceptual processes, which are hypothesized to be distributed relatively locally in the ventral visual stream, and emotion-valence processes, which are hypothesized to be distributed brain-wide and within a set of brain areas rather than voxels. 

To investigate these hypotheses, we want to use functional magnetic resonance imaging (fMRI) to measure and contrast neural representations of objects with neural representations of emotional valence. During fMRI acquisition, participants will be shown pictures of types of faces and two types of houses, embedded in an auditory narrative. Within the narrative, the pictures of neutral faces will be consistenly associated with either positive or negative valence, by characterizing the faces as the narrative's "hero" or "villain" respectively. The houses are     


Word count = 

%%%%%%%%%%%%%%%%%%%%%%%%%%%%%%%%%%%%%%%%
\section{Project description \textmd{- max 1200 words}}
\subsection{Prior research}
% Describe prior research, a comprehensible literature review of the research field, % converging upon the research questions. 
%
% a) Describe the state of affairs, including the theoretical framework, in the   current research field based on the existing body of literature.
% b) Clarify how the previous research eventuates into the research questions of the current proposal

% Example ref to test bib
\cite{barrett2006}

\subsection{Key questions}
% Now state the key questions, the essence of the proposal. Here, the intended research should be connected to prior research. Testable hypotheses should be derived from the key question, and the relation between theory and research hypotheses should be clearly specified.

% a) Formulate a general relevant research question based on previous research.
% b) Translate the general research question in a clear manner into a specific research question.
% c) Translate the specific research questions into testable research hypotheses.

Word count = 

%%%%%%%%%%%%%%%%%%%%%%%%%%%%%%%%%%%%%%%%
\section{Procedure \textmd{– approx. 1000 words}}
\subsection{Operationalisation}
% Describe how the research questions are operationalised. 
% a) Operationalise the research questions in a clear manner into a research design/strategy. 
% b) Describe the procedures for conducting the research and collecting the data. 

\subsection{Sample characteristics}
% a) Indicate, given a power analysis, how many participants will be recruited. Also motivate whether the resulting number is feasible.
% b) If a subset of participants will be excluded from the analysis given their scores on dependent variables, indicate the objective criterion to do so. For example include a phrase like: “Scores on dependent variables exceeding ± 3 sd of the mean will be excluded from the analysis’’
% c) If a subset of participants will be excluded from the analysis given their scores on a manipulation check item, indicate the objective criterion to do so. For example include a phrase like: “Participants scoring 15 or lower on a manipulation checks item, will be excluded from the analysis” 
% d) In case of a simulation study, indicate how data will be generated

\subsection{Materials}
% Indicate which tests, stimuli, equipment, etc. will be used; provide sufficiently elaborate descriptions and motivate your choice. (Always report the psychometric characteristics, such as reliability and validity, if existing tests are used. If new or adapted instruments or test materials (e.g., questionnaires) will be developed, then the new instrument must be independently validated first; only then it can be used as a testing instrument. Exception to this rule is allowed in case of questionnaires that do not contain more than one question (e.g., indicate on a 5-point scale how you feel today). 

\subsection{Data analysis}
% Indicate for each research question separately, how it is translated into a statistical prediction. For example: “In a repeated measures ANOVA we expect an interaction effect of the between factor x and the within factor y on the dependent variable z. Also indicate how you will correct for multiple comparisons. Only the analyses proposed here can be described as confirmatory analyses in your research report. All other have to be mentioned as exploratory. 

\subsection{Modifiability of procedure}
% Is there room for modification of the intended procedure? Evaluation of the proposal by the RMP Thesis Committee is meaningful only if the recommendations that the Committee might have can be implemented. It is therefore required that the intended procedure can be modified before you start gathering data. In situations where procedures or operationalisations or sample characteristics cannot be modified, the Thesis Committee has to be consulted before handing in the research proposal. The committee will consider the eligibility of this project for a research thesis. 

Word count =

%%%%%%%%%%%%%%%%%%%%%%%%%%%%%%%%%%%%%%%%
\section{Intended results \textmd{- max 250 words}}
% Clarify what the implication of possible outcomes would be (per hypothesis) for the specific and general research questions as well as for the theory. Address the following in approximately 250 words:

% a) What are the interpretations if the results do  match the expectations? 

% b) What are the interpretations if the results do not match the expectations?

% c) Are there any alternative interpretations?

% d) Is there any practical or societal relevance? Please explain. 

Word count = 

%%%%%%%%%%%%%%%%%%%%%%%%%%%%%%%%%%%%%%%%
\section{Work plan \textmd{– max 500 words}}
% Describe how the research project will be executed. Who is doing what and when? Is the planning of the current project realistic, efficient and feasible?

\subsection{Time schedule}
% State the total amount of ec as noted in the thesis contract (25-31 ec excl. proposal), 1 ec stands for 28 hours work. Present and justify a time schedule in weeks, including your time investment in hours per week. Plan some spare time, and indicate what elements can be cut / reduced if necessary.

\subsection{Infrastructure}
% Where will the research take place? How is access to the facilities and materials ensured?

\subsection{Budget}
% The compensation from the department is max € 80 for each research project. If the total expenditure exceeds the maximum compensation, then specify how the surplus will be financed. The € 80 budget may be used for printing costs (e.g. for the conference poster), travel expenses, participant payment. Specify the financial ramifications for the intended research. Please go to the secretariat of the specialization of your supervisor with your receipts. The secretariat will reimburse the costs you made up to € 80. 

Word count =

%%%%%%%%%%%%%%%%%%%%%%%%%%%%%%%%%%%%%%%%
% Bibliography, natbib with apalike style 
\renewcommand{\bibsection}{} % Turn off header creation
\section{References}
\bibliography{refs}
% List all cited literature, formatted according to the directions of the APA Manual.

%%%%%%%%%%%%%%%%%%%%%%%%%%%%%%%%%%%%%%%%
\section{Further steps}
Make sure your supervisor submits an Ethics Checklist for your intended research to the Ethics Committee of the Department of Psychology at http://ce.psy-uva.nl/. Submit the research proposal in PDF by email to researchmaster-psychology@uva.nl. 
If you have the proposal signed by the supervisor(s) and you have scanned their signatures in the PDF, you only have to hand in a digital version of the proposal. However if the signatures are not on the PDF, please also submit a printed copy of the signed research proposal to the secretariat of the Research Master Psychology:
\vspace{\baselineskip}

Universiteit van Amsterdam \\
Research Master’s Psychology \\
Weesperstraat 4, room 1.02 \\
1018 XA Amsterdam \\
researchmaster-psychology@uva.nl
\vspace{\baselineskip}

A response of the Research Master's Thesis Committee can be anticipated within 10 workdays (i.e. two weeks) after handing in the proposal. 

%%%%%%%%%%%%%%%%%%%%%%%%%%%%%%%%%%%%%%%%
\section{Signatures}
$\square$ I hereby declare that both this proposal, and its resulting thesis, will only contain original material and is free of plagiarism (cf. Chapter 11 or the Research Master's course catalogue).
\vspace{\baselineskip}

$\square$ I hereby declare that the result section of the thesis will consist of two subsections, one entitled "confirmatory analyses" and one entitled "exploratory analyses" (one of the two subsections may be empty):

\begin{enumerate}
\item The confirmatory analysis section reports exactly the analyses proposed in section 4 of this proposal
\item The exploratory analysis section contains additional, and thus exploratory, analyses. 
\end{enumerate}

\textbf{Signature student:} 
\vspace*{2\baselineskip} \\
\line(1,0){175} \\

\textbf{Signature ResMas supervisor:}
\vspace*{2\baselineskip} \\
\line(1,0){175} \\

% Uncomment if you have an external supervisor
%
%\textbf{Signature external supervisor (if applicable):}
%\vspace*{2\baselineskip} \\
%\line(1,0){175} \\

\end{document}